\newglossaryentry{gbtn}{		% unique label identifying this entry
	name=BaseTorrentNetzwerk,	% the name or term begin defined
	description={Peer2Peer Netzwerk aus Rechnern in verschiedenen Netzwerken}	% a description of this entry
}
\newacronym{btn}{BTN}{\glshyperlink{gbtn}}

\newglossaryentry{gbta}{
	name=BaseTorrentApplikation,
	description={Das Programm, welches den Zugang zum BTN ermöglicht und Benutzereingaben verarbeitet}
}
\newacronym{bta}{BTA}{\glshyperlink{gbta}}

\newglossaryentry{node}{
	name=Node,
	description={Im Allgemeinen eine andere Bezeichnung für einen Peer, die in den Vordergrund stellt, dass andere Peers über diesen Peer erreichbar sind}
}

\newglossaryentry{ip}{
	name=IP,
	description={Eine Adresse, die zur Identifikation im Netzwerk verwendet wird}
}

\newglossaryentry{xml}{
	name=XML,
	description={Eine spezielles Dateiformat mit <tag></tag> Struktur}
}

\newglossaryentry{udp}{
	name=UDP,
	description={Ein Protokoll zur Nachrichtenübertragung in einem Netzwerk}
}

\newglossaryentry{tcp}{
	name=TCP,
	description={Ein Protokoll zur Nachrichtenübertragung in einem Netzwerk; speziell zur Übertragung von großen Datenmengen geeignet}
}

\newglossaryentry{port}{
	name=Port,
	description={Ein Tor, um den richtigen Dienst auf dem Zielrechner anzusprechen}
}

\newglossaryentry{orgware}{
	name=Orgware,
	description={Rahmenbedingungen bei IT-Projekten, die nicht unter Hardware oder Software fallen}
}

\newglossaryentry{peer}{
	name=Peer,
	description={Ein Rechner im Netzwerk, der sowohl als Client als auch als Server fungiert}
}

\newglossaryentry{hash}{
	name=Hash,
	description={Eine Zahlen und Buchstabenfolge, die aus einer Datei zur Identifikation und Validierung gebildet wird}
}

\newglossaryentry{tracker}{
	name=Tracker,
	description={Ein Server-Programm, dass den Clients Informationen über die im BaseTorrentNetzwerk verteilten Dateien liefert, insbesondere den Aufenthaltsort der einzelnen Teile einer konkreten Datei}
}

\newglossaryentry{daten}{
	name=Daten,
	description={Dateien sowie ganze Ordner, die wiederum Ordner und Dateien enthalten können}
}

\newglossaryentry{Part-Passwort}{
	name=Part-Passwort,
	description={Bei \glshyperlink{backup}s wird neben dem Autorennamen im Klartext dieser zusätzlich nochmal in verschlüsselter Form gespeichert, um den Ersteller eindeutig zu identifizieren. Dieses Passwort wird Part-Password genannt und ist Teil der verschlüsselten \glshyperlink{meta}, aus denen nur der Inhaber das Passwort rekonstruieren kann}
}

\newglossaryentry{backup}{
	name=Backup,
	description={Sicherungskopie einer Datei oder eines ganzen Ordners}
}

\newglossaryentry{meta}{
	name=Meta-Daten,
	description={Informationsdaten über bestimmte \glshyperlink{daten}, z.B. "Teil x von y" oder der Dateiname}
}

\newglossaryentry{gbtf}{
	name=BaseTorrent file,
	description={Eine Datei (oder ein Ordner), die auf irgendeine Art und Weise im \glshyperlink{btn} bekannt ist}
}
\newacronym{btf}{BTF}{\glshyperlink{gbtf}}

\newglossaryentry{gbtfp}{
	name=BaseTorrent file part,
	description={Ein \glshyperlink{daten}-Teil bzw. Stück einer \glshyperlink{btf}}
}
\newacronym{btfp}{BTFP}{\glshyperlink{gbtfp}}

\newglossaryentry{peerliste}{
	name=Peerliste,
	description={Die Liste von allen \glshyperlink{node}s im \glshyperlink{btn}}
}

\newglossaryentry{redqua}{
	name=Redundanzqualit\"at, %\"a als da glossary kein utf8 unterstützt
	description={Die Redundanzqualität $k$ ist der Grad bzw. die Anzahl der Verteilungen eines \glshyperlink{btfp} im \glshyperlink{btn}. $k = min(\lceil 3\cdot\log\vert $\gls{peerliste}$\vert\rceil, \vert$\gls{peerliste}$\vert)$, d.h. ein fester Wert, damit alle Benutzer gleichberechtigt verteilen}
}

% \newglossaryentry{xxx}{
	% name=yyy,
	% description={zzz}
% }
