%%This is a very basic article template.
%%There is just one section and two subsections.
\documentclass[a4paper, ngerman]{scrartcl}

\usepackage[T1]{fontenc}
\usepackage[utf8]{inputenc}
\usepackage[ngerman]{babel}
\usepackage{lmodern}
\usepackage{amsmath}
\usepackage{amsfonts}
\usepackage{hyperref}
\usepackage{graphicx}
\usepackage{paralist}
\usepackage[none]{hyphenat}

\sloppy



\hypersetup{ 
pdfborder = {0 0 0}, 
urlbordercolor = {0 0 0}, 
colorlinks = true,
linkcolor = black,
citecolor = black,
filecolor = black,
urlcolor  = black
}

\title{Software-Challenge 2016 \\ Twixt} 
\subtitle{Spielregeln}



%% Variablen
\newcommand{\FelderAnzahl}{\emph{576}}
\newcommand{\EmptyPlainPage}{\newpage\thispagestyle{plain}\ \newpage}
\newcommand{\RundenAnzahl}{\emph{30}}

\begin{document} 
\parindent0px
\maketitle

\begin{figure}[h!]
	\centering
	\includegraphics[width=\linewidth]{bilder/gui.png} 
\end{figure}
\vspace*{\fill}
"Die Nutzung des Spielkonzeptes "twixt" (Name, Spielregeln
und Grafik) erfolgt mit freundlicher Genehmigung [verlag noch einfügen]."
\newpage
\tableofcontents
\newpage

\section{Einführung}
In dieser Anleitung werden die Elemente und Regeln des Spiels \emph{Twxit} der
Software-Challenge 2016 erläutert.\\
Zu Beginn des Spiels setzt zunächst Rot einen neuen Strommast auf ein freies
Feld, also ein Feld, was weder ein Sumpffeld noch ein Feld der gegnerischen
Farbe ist und auf dem sich noch kein Strommast befindet. Danach tut Blau
dasselbe, während beide versuchen ihre beiden Farbzonen mit einer Leitung zu
verbinden.

\section{Spielmaterial}
	\subsection{Das Spielbrett}
Das Spielbrett setzt sich aus 576 Bauflächen für Strommasten zusammen,
die entweder schon zu einer Farbe gehören, aber anfangs noch nicht bebaut sind,
von Sumpf bedeckt sind, also nicht bebaubar sind oder keines von beiden sind.
Anfangs werden zufällig Sumpfelder generiert, wobei es sich um einen kleinen
Sumpf (1x1), zwei mittlere Sümpfe (2x2) und einen großen Sumpf (3x3) handelt,
die sich auch überschneiden dürfen. Spielbrett ist in den Abbildungen 1-3 zu
sehen.
\section{Spielablauf}	 

	%% an Beispiel erklären
	Es beginnt der rote Spieler. Jeder Spieler darf abwechselnd einen
	Zug machen. Bei einem Zug ist es möglich, dass sich dadurch eine oder mehrere
	neue Leitungen bilden. Eine neue Leitung bildet sich, wenn sich ein eigener
	Strommast in folgendem Abstand zu dem gesetztem Strommast befindet: Zwei
	waagerecht und ein Feld senkrecht oder eins waagerecht und zwei Felder
	senkrecht. Dies geschieht aber nur, bisher keine Leitung existiert, die die
	entstehende Leitung kreuzen würde.
	Ein Beispiel welche Leitungen von einem Strommast aus möglich sind finden sie
	in Abbildung 1.
	 
	
	
	\subsection{Züge}
	
	\begin{figure}[h!]
		\centering
		\includegraphics[scale = 0.8]{bilder/setzzug.png}
		\caption{Alle möglichen Leitungen}
		\label{fig:Leitungen}
	\end{figure}
	%% an Beispiel erklären
	
\section{Ende des Spiels} 
	Das Spiel endet, sobald es einer der Spieler geschafft hat seine beiden
	Bereiche, die sich je nach Spieler oben und unten oder links und rechts
	befinden mit einer Leitung zu verbinden, spätestens jedoch nach 30 Runden.
	Die Punkteberechnung erfolgt wie folgt: Jedes Feld das von einem Startpunkt aus
	durch eine Leitung in die Spielrichtung (für Rot von links nach rechts
	beziehungsweise rechts nach links) überbrückt wird, zählt einen Punkt. Dies
	führt zu einer maximalen Punktzahl von 23 Punkten. Das Spiel gewinnt der
	Spieler mit der längsten Leitung in seine jeweilige Spielrichtung.
	Ist so ermittelte Punktzahl gleich ist das Ergebnis unendschieden.
	
\section{Die graphische Benutzeroberfläche}
\subsection{Graphische Benutzeroberfläche}
	In Abbildung~\ref{fig:GUI} ist ein Überblick der graphischen Benutzeroberfläche
	zu sehen. Die markanten Spielelemente sind mit \emph{A-C} gekennzeichnet.
	
	 \begin{figure}[h!]
		\centering		
		\includegraphics[scale = 0.6]{bilder/uebersicht.png} 
		\caption{Überblick der GUI}
		\label{fig:GUI}
	\end{figure} 
	%% wie ist die Benutzeroberfläche?
\begin{compactenum}[A)] 
\item Das Spielbrett
\item Die Spielfortschrittsanzeige  
\item Die Leitungslänge der Spieler
	\end{compactenum}
	
\subsection{Das Einstellungsmenü} 
	 \begin{figure}[h!]
		\centering
		\includegraphics[scale=0.6]{bilder/konfiguration.png}
		\caption{Das Einstellungsmenü}
		\label{fig:Configuration}
	\end{figure}
	
	Ein Einstellungsmenü mit Darstellungsoptionen wird in Abbildung 4 dargestellt
	und lässt sich über die Taste 'C' anzeigen. Dazu muss das
Spielfeld den Tastaturfokus haben (erforderlichenfalls
vorher Mausklick auf das Spielfeld). Es stehen dort
folgende Einstellungen zur Verfügung: 

\textbf{Kantenglättung} verbessert die Optik des
Spiels, ist aber rechenintensiv. Auf sehr langsamen Rechnern sollten sie daher
deaktiviert werden.
\textbf{Animationen} legt fest, ob die Bewegungen der Spielsteine in
Wiederholungen und bei Computerspielern animiert werden sollen.\\
Die \textbf{Debugansicht} zeigt Debug-Hilfestellungen zu einzelnen Zügen und
die Framerate an.
Diese Hilfestellungen sind Texte, die ein Spielclient einem Zug beifügen kann, den er
an den Spielserver sendet.
	
\end{document}
