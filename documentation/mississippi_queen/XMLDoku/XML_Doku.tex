\documentclass[12pt,a4paper, ngerman, oneside]{scrartcl}
\usepackage[ngerman]{babel}
\usepackage[utf8]{inputenc}
%\usepackage[T1]{fontenc}
%\usepackage{lmodern}
\usepackage{ucs}
\usepackage{amsmath}
\usepackage{amsfonts}
\usepackage{amssymb}
\usepackage{graphics}
\usepackage{paralist}
\usepackage[linkcolor=black]{hyperref}
\usepackage{listings} \lstset{numbers=left, numberstyle=\tiny, numbersep=5pt} \lstset{language=Java}
%% Grafiken
\usepackage[pdftex]{graphicx}
\usepackage{epsfig} 
\hypersetup{% 
  colorlinks=true,
}
\sloppy
\hyphenpenalty=100000

\date{Software-Challenge Germany 2017\\Stand \today}


%\author{Sören Domrös, stu114053@mail.uni-kiel.de }
\title{Client-Server Kommunikation Mississippi Queen}


\begin{document}
\maketitle
\thispagestyle{empty}
\tableofcontents
\thispagestyle{empty}
\newpage
\setcounter{page}{1}
\section{Einleitung}
\subsection*{Zum Inhalt} Wie in den letzten Jahren wird zur Client-Server
Kommunikation ein XML-Protokoll genutzt\footnote{siehe auch:
\htmladdnormallink{Allgemeine Dokumentation: Die Schnittstelle zum
Server}{https://cau-kiel-tech-inf.github.io/socha-enduser-docs/\#schnittstelle}}. In diesem
Dokument wird die Kommunikationsschnittstelle definiert, sodass ein komplett
eigener Client geschrieben werden kann. Es wird hier nicht die vollständige
Kommunikation dokumentiert bzw. definiert, dennoch alles, womit ein Client
umgehen können muss, um spielfähig zu sein.
\subsection*{An wen richtet sich dieses Dokument?} Die Kommunikation mit dem
Spielserver ist für diejenigen, die aufbauend auf dem Simpleclient
programmieren, unwichtig. Dort steht bereits ein funktionierender Client bereit
und es muss nur die Spiellogik entworfen werden. \\
Nur wer einen komplett eigenen Client entwerfen will, beispielsweise um die
Programmiersprache frei wählen zu können, benötigt die Definitionen.

\subsection*{Hinweise} Falls Sie beabsichtigen sollten, diese
Kommunikationsschnittstelle zu realisieren, sei darauf hingewiesen, dass es im
Verlauf des Wettbewerbes möglich ist, dass weitere, hier noch nicht aufgeführte
Elemente zur Kommunikationsschnittstelle hinzugefügt werden. Um auch bei
solchen Änderungen sicher zu sein, dass ihr Client fehlerfrei mit dem Server
kommunizieren kann, empfehlen wir Ihnen, beim Auslesen des XML jegliche Daten zu
verwerfen, die hier nicht weiter definiert sind. \bigskip \\
Die vom Institut bereitgestellten Programme (Server, Simpleclient) nutzen eine
Bibliothek um Java-Objekte direkt in XML zu konvertieren und umgekehrt. 
Dabei werden XML-Nachrichten nicht mit einem newline abgeschlossen.
%Dabei
%werden den XML-Tags jeweils \textit{id}-Attribute beigefügt. Auch diese können
%vernachlässigt werden.

\subsection*{Wie das Dokument zu lesen ist}
Es finden sich für einzelne Arten von Nachrichten kurze Definitionen. Dabei ist ein allgemeiner XML-Code gegeben, bei dem Attributwerte durch Variablen ersetzt sind, die über dem Code kurz erläutert werden.\\
Eingebettete XML-Elemente werden hier allgemein als \begin{verbatim}
{TAGNAME}
\end{verbatim} geschrieben, wenn an der Stelle eine beliebige Anzahl von \textit{TAGNAME}-Tags eingebettet sein \textit{kann} und als \begin{verbatim}
[TAGNAME]
\end{verbatim}, wenn an der Stelle ein \textit{TAGNAME}-Tag optional stehen kann. \\

\subsection{Beispiel-Definition}
Die Definitionen von a- und b-Tags unten lassen unter anderem folgende a-Tags zu:
\begin{verbatim}
- <a name="Sinnvoll">
  </a>
- <a name="Sinnvoll">
      <b anzahl="2"/>
  </a>
- <a name="Hans">
      <b anzahl="2"/>
      <b anzahl="2"/>
      <b anzahl="2"/>
  </a>
\end{verbatim}
\subsubsection*{Definition 'b'}
\begin{verbatim}
<b anzahl="2">
\end{verbatim}
\subsubsection*{Definition 'a'}
\begin{description}
\item[N] Ein beliebiger Name
\item[b] ein b-Tag wie in obiger Definition
\end{description}
\begin{verbatim}
<a name="N" >
	{b}
<\a>
\end{verbatim}

\newpage
\part{Client $\rightarrow$ Server}
\section{Spiel betreten}
\subsection{Ohne Reservierung}
Betritt ein beliebiges offenes Spiel:
\begin{verbatim}
<join gameType="swc_2017_mississippi_queen"/>
\end{verbatim}
\subsection{Mit Reservierungscode}
Ist ein Reservierungscode gegeben, so kann man den durch den Code gegebenen Platz betreten.
\begin{description}
\item[RC] Reservierungscode
\end{description}
\begin{verbatim}
<joinPrepared reservationCode="RC"/>
\end{verbatim}

\section{Züge senden}

\subsection{\label{Move}Der Move}
Der Move ist der allgemeine Zug, der in verschiedenen Varianten gesendet werden kann.
\begin{description}
\item[RID] ID des Raumes
\item[ZUG] Informationen über den Zug
\begin{verbatim}
<room roomId="RID">
  ZUG
</room>

\end{verbatim}
\subsection{Zug}
\label{move}
\begin{description}
\item[ACTION] eine Teilaktion des Zuges
\end{description}
\begin{verbatim}
	<data class="move">
	  <actions>
	    ACTION
	    ..
	    ACTION
	  </actions>
	</move>
\end{verbatim}

\subsection{ACTION}
\label{action}
\begin{description}
\item[I] Nummer der Aktion
\item[M] Beschleunigung M $\in \{-5,-4..5\}$
\item[N] Anzahl der Drehungen N $\in \{-3..3\}$ negativ für gegen, positiv für mit dem Uhrzeigersinn
\item[V] Anzahl der Felder die in Bewegungsrichtung zurückgelegt werden (-1 für Rückwärts)
\item[D] Richtung in die abgedrängt wird. D $\in \{0..5\}$
\end{description}
\begin{verbatim}
	<acceleration order="I" acc="M"/>
	<turn order="I" direction="N"/>
	<step order="I" distance="V"/>  
	<push order="I" direction="D"/>
\end{verbatim}
\end{description}

\subsection{Debughints}
Zügen können Debug-Informationen beigefügt werden:
\begin{description}
\item[S] Informationen zum Zug als String
\end{description}
\begin{verbatim}
 <hint content="S"/>
\end{verbatim}
Damit sieht beispielsweise ein Laufzug so aus:
\begin{verbatim}
<room roomId="RID">
  <data class="move">
    <actions>
      <acceleration order="0" acc="1"/>
      <step order="1" distance="3"/>
      <turn order="2" direction="-1"/>
      <step order="3" distance="1"/>
    </actions>
    <hint content="Dies ist ein Hint."/>
    <hint content="noch ein Hint"/>
  </data>
</room>
\end{verbatim}



\newpage
\part{Server $\rightarrow$ Client}
\begin{description}
\item[RID] ID des Raumes, in dem das Spiel stattfindet
\end{description}

\section{Raum beigetreten}
 \begin{verbatim}
 <joined roomId="RID"/>
 \end{verbatim}

\section{Willkommensnachricht}
Der Spieler erhält zu Spielbeginn eine Willkommensnachricht, die ihm seine Farbe mitteilt.
\begin{description}
\item[C] Spielerfarbe (red/blue)
\end{description}
\begin{verbatim}
<room roomId="RID">
  <data class="welcomeMessage" color="C"/>
</room>
\end{verbatim}

\section{Spielstatus}
Es folgt die Beschreibung des Spielstatus, der vor jeder Zugaufforderung an die Clients gesendet wird. Das Spielstatus-Tag ist dabei noch in einem \textit{data}-Tag der Klasse \textit{memento} gewrappt:
\subsection{memento}
\begin{description}
\item[status] wie in \ref{state} definiert
\end{description}
\begin{verbatim}
<room roomId="RID"> 
  <data class="memento"> 
  	status
  </data> 
</room>
\end{verbatim}

\subsection{\label{state}Status}
\begin{description}
\item[Z] aktuelle Zugzahl
\item[S] Spieler, der das Spiel gestartet hat (RED/BLUE)
\item[C] Spieler, der an der Reihe ist (RED/BLUE)
\item[red, blue] wie in \ref{player} definiert
\item[board] Das Spielbrett, wie in \ref{board} definiert
\item[lastMove] Letzter getätigter Zug (nicht in der ersten Runde), wie in
\ref{lastmove} definiert
\item[condition] Spielergebnis, wie in \ref{gameend} definiert; nur zum Spielende
\end{description}
\begin{verbatim}
<state class="state" turn="Z" startPlayer="S" currentPlayer="C">
  red
  blue
  [board]
  [lastMove]
  [condition]
</state>

\end{verbatim}

\subsection{\label{player}Spieler}
\begin{description}
\item[C] Farbe (RED/BLUE)
\item[N] Anzeigename
\item[P] Punktekonto
\item[X] X Koordinate des Spielers
\item[Y] Y Koordinate des Spielers
\item[D] Ausrichtung des Spielers, 0 rechts, Rest mit dem Uhrzeigersinn
\item[S] Geschwindigkeit des Spielers
\item[K] Kohleeinheiten des Spielers
\item[T] Teilabschnitt, auf dem sich der Spieler befindet
\item[PA] Anzahl der Passagiere
\end{description}
\begin{verbatim}
<C displayName="N" color="C" points="P" x="X" y="Y" 
direction="D" speed="S" coal="K" tile="T" passenger="PA"/>
\end{verbatim}


\subsection{\label{board}Spielbrett}
\begin{description}
\item[FIELD] Ein Spielfeld wie in \ref{sec:field} definiert.
\end{description}
\begin{verbatim}
<board>
 <tiles>
   <tile index="0" direction="0">
 	{FIELD}
 	..
 	{FIELD}
   </tiles>
   ..
 </tiles>
</board>
\end{verbatim}

\subsection{\label{sec:field}Spielfeld}
\begin{description}
\item[Type] Typ des Feldes (WATER/BLOCKED/PASSENGER{i}/GOAL/SANDBAR/LOG). ($i \in \{0..5\}$ gibt die Richtung an)
\item[X] X-Koordinate des Feldes
\item[Y] Y-Koordinate des Feldes
\end{description}
\begin{verbatim}
<field type="Type" x="X" y="Y"/>
\end{verbatim}

\subsection{\label{lastmove}Letzter Zug}
Der letzte Zug ist ein Move (siehe hierzu \ref{move}).
\begin{description}
\item[ACTIONS] Teilzüge des Zuges (siehe hierzu \ref{action}).
\end{description}
\begin{verbatim}
<lastMove>
  ACTIONS
</lastMove>
\end{verbatim}

\section{\label{moverequest}Zug-Anforderung}
Eine einfache Nachricht fordert zum Zug auf:
\begin{verbatim}
<room roomId="RID">
  <data class="sc.framework.plugins.protocol.MoveRequest"/>
</room>
\end{verbatim}

\section{Fehler}
Ein \glqq spielfähiger\grqq\ Client muss nicht mit Fehlern umgehen können.
Fehlerhafte Züge beispielsweise resultieren in einem vorzeitigen Ende des
Spieles, das im nächsten gesendeten Gamestate erkannt werden kann (siehe \ref{gameend}).
\begin{description}
\item[MSG] Fehlermeldung
\end{description}
\begin{verbatim}
<room roomId="RID">
	<error message="MSG">
		<originalRequest>
		Request, der den Fehler verursacht hat
		</originalRequest>
	</error>
</room>
\end{verbatim}

\section{Spiel pausiert}
Ein \glqq spielfähiger\grqq\ Client muss Pausierungsnachrichten nicht beachten,
da er nur auf Aufforderungen (Zug-Aufforderung siehe \ref{moverequest}) des Servers handelt.
\begin{description}
\item[C] Farbe (RED/BLUE)
\item[N] Anzeigename
\item[P] Punktekonto
\item[X] X Koordinate des Spielers
\item[Y] Y Koordinate des Spielers
\item[D] Ausrichtung des Spielers 0 rechts Rest mit dem Uhrzeigersinn
\item[S] Geschwindigkeit des Spielers
\item[K] Kohleeinheiten des Spielers
\item[T] Spielabschnitt auf dem sich der Spieler befindet
\item[PA] Anzahl der Passagiere
\end{description}
\begin{verbatim}
<room roomId="RID">
	<data class="paused">
		<nextPlayer class="player" displayName="N" color="C" points="P" x="X" y="Y" 
direction="D" speed="S" coal="K" tile="T" passenger="PA"/>
	</data>
</room>
\end{verbatim}

\section{Spiel verlassen}
Wenn ein Client den Raum verlässt, bekommen die anderen Clients eine entsprechende Meldung vom Server.
\begin{verbatim}
<left roomId="RID"/>
\end{verbatim}


\section{\label{gameend}Spielergebnis}
Zum Spielende enthält der Spielstatus eine \textit{Condition}, der das Spielergebnis entnommen werden kann:
\begin{description}
\item[W] Gewinner (RED/BLUE), bei Unentschieden wird kein winner mitgeschickt.
\item[R] Text, der den Grund für das Spielende erklärt
\end{description}
\begin{verbatim}
<condition winner="W" reason="R"/>
\end{verbatim}

\newpage
\part{Überblick}
Hier nochmal ein kurzer Überblick, der etwas genauer zeigt, wie die
Kommunikation ablaufen kann.\bigskip\\

\begin{enumerate}
\item Ein Spielserver startet ein Spiel und wartet auf den Client (die Clients).
\item Der Client stellt eine TCP Verbindung zum Spielserver her (er kennt dessen IP-Adresse und Port)
\item Die Verbindung ist aufgebaut und der Client sendet \begin{verbatim}
<protocol>
    <join gameType="swc_2017_mississippi_queen"/>
\end{verbatim}
\item Der Server sendet \begin{verbatim}
<protocol>
    <joined roomId="65d3d704-87d9-42eb-8995-51c3e6324513"/>
\end{verbatim}
\item der Client merkt sich die roomId.
\item der Server startet das Spiel und sendet \begin{verbatim}
<room roomId="65d3d704-87d9-42eb-8995-51c3e6324513">
    <data class="welcomeMessage" color="blue"/>
</room>
\end{verbatim}
\item der Client merkt sich seine Spielerfarbe.
\item der Server sendet das Ausgangsspielfeld: 
\begin{verbatim}
<room roomId="65d3d704-87d9-42eb-8995-51c3e6324513">
  <data class="memento">
    <state class="state" turn="0" startPlayer="RED" currentPlayer="RED">
      <red displayName="Spieler 1" color="RED" points="0" x="-1" y="1"
       direction="0" speed="1" coal="6" tile="0" passenger="0"/>
      <blue displayName="Spieler 2" color="BLUE" points="0" x="-1" y="-1"
       direction="0" speed="1" coal="6" tile="0" passenger="0"/>
      <board>
        <tiles>
          <tile index="0" direction="0">
            <field type="WATER" x="0" y="19"/>
            <field type="WATER" x="0" y="20"/>
            <field type="WATER" x="0" y="21"/>
            <field type="WATER" x="0" y="22"/>
            <field type="WATER" x="0" y="23"/>
            ...
           </tile>
           ..
           <tile index="7" direction="3">
            ...
            <field type="WATER" x="34" y="5"/>
            <field type="TARGET" x="34" y="6"/>
            <field type="TARGET" x="34" y="7"/>
            <field type="TARGET" x="34" y="8"/>
            <field type="WATER" x="34" y="9"/>
          </tile>
       <fields>
     </board>
    </state>
  </data>
</room>
\end{verbatim}
\item da der andere Spieler anfängt folgt noch eine \textit{memento}-Nachricht vom Server, die den Spielstatus nach dem ersten Zug von Spieler 1 beinhaltet.
\item es folgt die erste Zug-Aufforderung vom Server: \begin{verbatim}
<room roomId="65d3d704-87d9-42eb-8995-51c3e6324513">
    <data class="sc.framework.plugins.protocol.MoveRequest"/>
</room>
\end{verbatim}
\item Der Client antwortet mit einem Zug: \begin{verbatim}
<room roomId="65d3d704-87d9-42eb-8995-51c3e6324513">
  <data class="move">
      <actions>
    	  <acceleration order="0" acc="2"/>
    	  <step order="1" distance="3"/>
      </actions>
  </data>
</room>
\end{verbatim}
\item so geht es weiter...
\item die letzte Nachricht enthält ein \begin{verbatim}
</protocol>
\end{verbatim}
Daraufhin wird die Verbindung geschlossen
\end{enumerate}

\end{document}
